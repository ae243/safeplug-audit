\section{Procedure}
\label{sec:proc}
Although Safeplug uses open source components (not that they are properly documented, as we discovered in Section \ref{tos} below), the device itself is not open source, which makes a security audit much more challenging.  We approached this audit both from the perspective of a user (see Section \ref{sec:ux}) and the perspective of a security researcher (Section \ref{sec:tear} - \ref{sec:SSH}).  This included a teardown of one of the boxes to identify components, capturing network traffic, and accessing the box via SSH.

We set up a mock ``home network'' such as the device might be expected to see and then capture traffic both inside that network and between the network and the internet.  We did this by using a computer with two network cards.  One card was connected to a Netgear switch and served as a gateway, DHCP server, and DNS server.  The other card was the connection to the internet.  Setting up the various components, as well as a bridge and IP forwarding across the two network cards was a particularly time-consuming endeavour, but it allowed us to capture Safeplug traffic as it might occur in the wild.  We used Wireshark to analyze the network data that we captured \cite{wireshark}.  We are interested both in the communication between the clients and the Safeplug and between the Safeplug and the outside world. 

We expected the SSH component of the project to be a challenge, but thanks to Pogoplug's lack of security, the instructions and root password were available online!  We analyzed the filesystem on the Safeplug discovered an RPC server that was subject to attack from insiders and potentially from external threats as well.
