\section{Introduction}
\label{sec:intro}

User privacy is becoming increasingly important as more Internet users realize how vulnerable they are to attackers and eavesdroppers.  This has been exacerbated by the current news about the NSA's surveillance of the Internet~\cite{nsa}.  Unfortunately, the average Internet user is not aware of how to protect themselves against malicious users.     

{\bf Tor.} The state-of-the-art in online anonymity technology is the open-source project Tor.  Tor is a service that provides anonymous communication as an onion router; by encrypting internet traffic and sending it through layers of relays, a user can make it much more difficult to trace their internet use~\cite{tor}.  Instead of appearing to come from a user's IP address, the traffic will reach the destination from one of the relays, that will have received the traffic from another relay, creating a chain back from when the user first sent their traffic to a relay.  Tor is known as an onion router because each relay peels back one layer of encryption and no more, allowing it to send the traffic to the next destination but not allowing it to discover the origin or final recipient of the traffic.  Tor is open-source software developed by the Tor Project~\cite{torproject}.  Users of Tor are highly recommended to use it as part of the {\bf Tor Browser Bundle (TBB)}, which provides a single installation of a browser and the Tor software package.  This browser has special settings to prevent deanonymization of the traffic by other means, such as cookies, supercookies, or scripts~\cite{torproject}.  
Using Tor or the Tor Browser Bundle provides a tradeoff between anonymity, usability, and efficiency.  Sending traffic to these relays around the world slows down the traffic significantly, possibly degrading user experience for websites that load a lot of data or function in real-time.  In addition, because the destination website will see the user's request as coming from a new, likely international IP address, websites such as banks that have location-based safeguards may deny users access to their sites.  For many users around the world, these troubles are a worthwhile price for access to free and open internet, anonymous communications, and resisting censorship.  However, one of the biggest reasons that Tor is not more widespread is that many Internet users do not know about Tor, how it works, or how they would be able to use it to become more anonymous online.

{\bf Safeplug.} Safeplug is a product that launched in December of 2013 from the cloud storage company Pogoplug, that offers any user the option of using Tor without having to know about it or how it works.  It allows users to browse the web from their own standard web browser with “complete security and anonymity” for the cheap price of \$49 \cite{safeplug}.  Safeplug offers Tor out of the box, with no additional software installation, by sitting between a user’s router and the internet \cite{wired}.  Pogoplug’s marketing pitch centers around the protection of user’s IP addresses by using Tor \cite{safeplug,bittech}.

{\bf Pogoplug.} Pogoplug, a subsidary of Cloud Engines, also offers a box for home network file sharing without needing to use a third-party storage provider.  Based on our examination of the box discussed below, we suspect that the Pogoplug box was relabelled as Safeplug and just uses different software to achieve its newly branded purpose.
