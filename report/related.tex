\section{Related Work}
\label{sec:related}

To our knowledge, there has been no other study analyzing the security of Pogoplug's Safeplug device.  However there has been much prior research on the primary technology that the device uses: Tor.  Additionally, the security goals that Safeplug attempts to achieve have been previously studied in great detail; these include fingerprinting and cookies.  

{\bf Tor.} Prior security evaluations of the Tor network reveal a myriad of potential vulnerabilities.  A significant area of research on Tor relates to diversity of autonomous systems (ASes).  Feamster and Dingledine argue that a user's anonymity may be compromised by using geographically diverse ASes; when analyzing both sides of an anonymous path, it is more likely the same AS will appear in both sides of a long path than in a short path~\cite{feamster}.  Murdoch and Zielinski also argue against AS diversity.  They state that AS diversity does not improve security because traffic is routed through ASes at Internet exchange points (IXPs); therefore, an IXP can observe traffic that passes through multiple ASes~\cite{murdoch2}.  There has also been proven traffic correlation attacks that are efficient on the Tor network~\cite{murdoch, overlier}.  Johnson, Wacek, Jansen, Sherr, and Syverson evaluated Tor's security against reasonably realistic adversaries and contribute metrics that model security over time.  They found that in a period of six months, 80\% of all users may be deanonymized by a reasonably realistic Tor-relay adversary.  Johnson et al. take into account how the Tor network evolves over time when evaluating Tor's security~\cite{tor2}.  While Safeplug does not introduce or modify how Tor is used, it routes all traffic through the Tor network; Safeplug is also vulnerable to the attacks found in prior research on Tor.  

{\bf Fingerprinting.}  Website fingerprinting attacks as well as remote physical device fingerprinting attacks have shown they can identify users, even when specific defense have been used in order to prevent these attacks. Previous research has shown that web page fingerprinting attacks are possible~\cite{dyer, herrmann, panchenko}.  Cai, Zhang, Joshi, and Johnson introduced both a web page and website fingerprinting attack that defeats many recently proposed defenses, including cover traffic and randomized pipeling.  They find that their attack is successful 83.7\% of the time when the defense is the use of Tor; the success rate decreases to 52.2\% when the defense is the use of Tor, randomized pipelining, padding packets to 1500 bytes, and adding cover traffic at a 1:1 ratio~\cite{fingerprint1}.  These results can be extended to the security of Safeplug.  Because Safeplug uses only Tor to anonymize users, it may be susceptible to this fingerprinting attack.

{\bf Cookies} There has been much policy and technology debate around the topic of third-party web tracking.  Mayer and Mitchell survey recent policy and technology stances on web tracking; while tracking allows for free web content and innovation, it compromises a user's privacy~\cite{commercial2}.  They discuss the existence and use of stateful tracking by using ``supercookies.''  While many companies use cookies, there has also been work in developing privacy-preserving third-party services, such as Privad, Adnostic, and RePriv~\cite{guha, toubiana, fredrikson}.  
