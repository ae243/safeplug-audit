\section{Implementing Attacks}
\label{sec:implementattacks}

\subsection{Gaining Access through SSH}

\subsection{Insider Attack}

\subsection{CSRF Attack}
An external website can also perform this attack by returning a correctly formatted POST string.  This executes the same functionality as the Insider Attack, but the attacker does not need to be on the local network or know the IP address of the Safeplug.  Instead, the attacker (who could be any malicious actor with access to a web server) can just send a POST request to every IP in the common ranges of local addresses in home networks, 192.168.0.0/24 and 192.168.1.0/24.  We implemented this attack with less than 30 lines of Javascript code (See Appendix A).  The following steps are necessary for the attack to be successful:

\begin{enumerate}
\item Set up a web page with the crafted Javascript code, which will send a POST request of the following format to all addresses in the range: http://$<$IP address$>$/svc/xspctrl/disableTor (See Appendix A).
\item Send the malicious link to a user in the targeted private network.
\item Once the user clicks the link and loads the malicious site, the correctly formatted POST request will be sent to every IP address in the ranges.  
\item Tor is disabled silently.  The user must check or refresh her settings page to learn that Tor is turned off.  
\end{enumerate}  

This exploits the RPC server in the same manner that the Insider Attack does.  While this attack requires a greater amount of time because the local IP address of the Safeplug must be guessed via search, but the number of private address spaces is small, and the space likely to be occupied by a Safeplug on a home network is even smaller.  

The largest observed time to send requests to the 192.168.0.0/24 space was ~400 milliseconds; the entire attack costs ~800 milliseconds for sending request to both 192.168.0.0/24 and 192.168.1.0/24 ranges - even when the website was being loaded over Tor.  In the case of a private network in the range of 172.16.0.0/16, the attack took less than ~12 minutes (this generates script timeout warnings in most major browsers, which affects the timing of this attack).  This means that it would take a few hours to send requests to the full 172.16.0.0/12 range, which is commonly used in business networks.  The final private network space is 10.0.0.0/8 which is too large for an exhaustive search, but some simple optimization might make it feasible as well.  For example, using a GET request to get and parse the Safeplug settings page would allow the script to positively identify the safeplug and stop the search.  However, the 192.168.0.0/24 and 192.168.1.0/24 ranges are much more common in home networks; because Safeplug is geared towards home network use, in most cases the script will take less than a second, a trivial amount of time for the attacker to spend.  

In addition to disabling Tor, the attacker can modify any other settings on the device.  This includes: enabling/disabling Tor, enabling/disabling ad-blocking, enabling/disabling the use of the device as a Tor relay node [note: this requires the user to do additional setup], enabling/disabling the use of the device as an exit node (if it is already a relay).  Lastly, the attacker can also modify the user's whitelist of sites that should not be routed through Tor.  This whitelist attack is particularly dangerous because the change is silent and much harder for the user to notice the addition of a single website to the whitelist than a global loss of Tor.  (Removal of a webpage from the whitelist is likely to to cause usability problems and be more evident.)
