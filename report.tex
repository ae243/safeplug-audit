\documentclass[12pt, letterpaper]{article}
\usepackage{url}

\title{Security Audit of Safeplug}
\author{Annie Edmundson and Anna Kornfeld Simpson}

\begin{document}

\maketitle

\section{Introduction}

\section{Background}
Safeplug is a product that launched last week from the cloud storage company Pogoplug, offering to let users browse the web with “complete security and anonymity” for the cheap price of \$49 \cite{safeplug}.  Safeplug offers Tor out of the box, with no additional software installation, by sitting between a user’s router and the internet \cite{wired}.  Pogoplug’s marketing pitch centers around the protection of user’s IP addresses by using Tor \cite{safeplug,bittech}.  Lots of folks are excited about the idea, but waiting to see a security audit, including Bruce Schneier \cite{schneier}.  Gigaom quotes Pogoplug CEO Dan Putterman as saying the box uses “vetted software” \cite{gigaom} but who knows what that means?

\section{Goals}
In this project, we would like to do a security audit of the Safeplug and investigate the following questions.  How does the box pick which relay node on the Tor network to connect to?  What kind of authentication goes on between the box and the Tor node?  Does it use the same route of Tor relays every time?
Safeplug claims to offer in-box ad-blocking.  How does this work and does it remove any functionality?  Is there an easy way to whitelist things without turning Safeplug off entirely?

What are the configurations on the box?  Is it clear to users what information is being hidden and what information is being leaked?  The website talks a lot about IP addresses and warns users to clear cookies \cite{safeplug}, but is that sufficient to preserve anonymization across all web traffic (including Flash or Javascript supercookies that might not be cleared when cookies are cleared)?  Can user anonymity be leaked by a user’s behavior while still using the Safeplug?  What else must a user do in addition to using the Safeplug to further preserve anonymity?

Safeplug can act as a relay in the Tor network \cite{techreview}; how would widespread use of the box affect the Tor network?

\section{Methodology}
Safeplug is not open source, which makes a security audit much more challenging.  This leads to a debate on where a user will put his/her trust.  Despite not having the source code, we hope to get a Safeplug, set it up via their website, and use Wireshark to capture traffic and see what happens.  Wireshark allows us to capture and analyze network data \cite{wireshark}.  We are interested both in the communication between the clients and the Safeplug and between the Safeplug and the outside world.  

\section{Results}

\subsection{Comparison with Tor Browser Bundle}

\section{Conclusion}

\begin{thebibliography}{99}

\bibitem{spymall} Appelbaum et. al. ``Shopping for Spy Gear: Catalog Advertises 
NSA Toolbox'' \emph{Der Spiegel} 29 December 2013 \url{www.spiegel.de/internatio
nal/world/catalog-reveals-nsa-has-back-doors-for-numerous-devices-a-940994.html}

\bibitem{nsa} Arther, Charles.  ``NSA scandal: what data is being monitored and how does it work?'' \url{http://www.theguardian.com/world/2013/jun/07/nsa-prism-records-surveillance-questions}.

\bibitem{fingerprint1} Cai, Xiang, et al. ``Touching from a distance: Website fingerprinting attacks and defenses.'' \emph{Proceedings of the 2012 ACM conference on Computer and Communications Security (CCS)}, 2012.

\bibitem{ceadmin} Colleton, Lee. ``Fwd: SSH on Safeplug'' \emph{Tor-talk mailing list} 2 January 2014. \url{http://archives.seul.org/or/talk/Jan-2014/msg00003.html}

\bibitem{tor} Dingledine, Roger, et. al. ``Tor: The second-generation onion router.'' \emph{USENIX Security Symposium}, 2004.

\bibitem{dyer} Dyer, Kevin P., Coull, Scott E., Ristenpart, Thomas, and Shrimpton, Thomas.  ``Peek-a-boo, i still see you: Why efficient traffic analysis countermeasures fail.'' \emph{In Proceedings of the 33rd Annual IEEE Symposium on Security and Privacy}, 2012.

\bibitem{feamster} Feamster, Nick, and Dingledine, Roger.  ``Location Diversity in Anonymity Networks.''  \emph{In ACM Workshop on Privacy in the Electronic Society (WPES)}, 2004.

\bibitem{fredrikson} Fredrikson, M., and Livshits, B. ``Repriv: Re-envisioning in-browser privacy.'' \emph{In Proceedings of the 2011 IEEE Symposium on Security and Privacy}, May 2011.

\bibitem{guha} Guha, S., Cheng, B., and Francis, P. ``Privad: Practical privacy in online advertising.'' \emph{In Proceedings of the 2011 USENIX Symposium on Networked Systems Design and Implementation}, April 2011.

\bibitem{bittech} Halfacree, Gareth. ``Pogoplug launches Tor-powered Safeplug'' \emph{bit-tech.net} 25 November 2013, \url{http://www.bit-tech.net/news/hardware/2013/11/25/pogoplug-safeplug/1}.

\bibitem{herrmann} Herrmann, Dominik, Wendolsky, Rolf, and Federrath, Hannes.  ``Website fingerprinting: attacking popular privacy enhancing technologies with the multinomial naive-bayes classifier.'' \emph{In Proceedings of the 2009 ACM workshop on Cloud computing security}.

\bibitem{tor2} Johnson, Aaron, et al. ``Users get routed: Traffic correlation on Tor by realistic adversaries.'' \emph{Proceedings of the 2013 ACM SIGSAC conference on Computer \& Communications Security (CCS)}, 2013.

\bibitem{dropbear} Johnston, Matt. ``Dropbear SSH.'' \url{https://matt.ucc.asn.au/dropbear/dropbear.html}

\bibitem{fingerprint2} Kohno, Tadayoshi et. al. "Remote physical device fingerprinting." Dependable and Secure Computing, IEEE Transactions on 2.2 (2005): 93-108.

\bibitem{lighttpd} Lighttpd. \url{http://www.lighttpd.net}


\bibitem{marvellhw} Marvell, ``88F6190 and 88F6192 Integrated Controller Hardware Specifications'' 2 December 2008. \url{http://www.marvell.com//embedded-processors/kirkwood/assets/HW_88F619x_OpenSource.pdf}

\bibitem{marvell} Marvell, ``Marvell 88F6192 SoC with Sheeva Technology'' 2009. \url{http://www.marvell.com//embedded-processors/kirkwood/assets/88F6192-003_ver1.pdf}

\bibitem{marvell2} Marvell, ``Marvell Alaska 88E1116R'' 2007. \url{http://www.marvell.com/transceivers/assets/Marvell-Alaska-88E1116R-Single-Port-GbE.pdf}

\bibitem{fourthparty} Mayer, Jonathan. ``FourthParty'' \url{http://fourthparty.info}

\bibitem{commercial2} Mayer, Jonathan R., and John C. Mitchell. "Third-party web tracking: Policy and technology." \emph{IEEE Symposium on Security and Privacy (SP)}, 2012.

\bibitem{gigaom} Meyer, David. ``Say hello to Safeplug, Pogoplug’s \$49 Tor-in-a-box for anonymous surfing'', \emph{GigaOm}, 21 November 2013. \url{http://gigaom.com/2013/11/21/say-hello-to-safeplug-pogoplugs-49-tor-in-a-box-for-anonymous-surfing/}.

\bibitem{murdoch} Murdoch, S.J., Danezis, G.  ``Low-Cost Traffic Analysis of Tor.'' \emph{In IEEE Symposium on Security and Privacy (Oakland)}, 2005.

\bibitem{murdoch2} Murdoch, S.J. and Zielinski, P.  ``Sampled Traffic Analysis by Internet-Exchange-Level Adversaries.'' \emph{In Privacy Enhancing Technologies (PET)}, 2007.

\bibitem{overlier} Overlier, L. and Syverson, P.  ``Locating Hidden Servers.'' \emph{In IEEE Symposium on Security and Privacy (Oakland)}, 2006.

\bibitem{panchenko} Panchenko, Andriy, Niessen, Lukas, Zinnen, Andreas, and Engel, Thomas.  ``Website fingerprinting in onion routing based anonymization networks.'' \emph{In Proceedings of the 10th Workshop on Privacy in the Electronic Society}, 2011.

\bibitem{pano} Panopticlick. \url{https://panopticlick.eff.org/}.

\bibitem{safeplug} Pogoplug. ``Safeplug'', \url{https://pogoplug.com/safeplug}.

\bibitem{privoxy} Privoxy. \url{https://www.privoxy.org}

\bibitem{schneier} Schneier, Bruce. ``Tor Appliance.'' \emph{Schneier on Security}, 27 November 2013. \url{https://www.schneier.com/blog/archives/2013/11/tor_appliance.html}.

\bibitem{techreview} Simonite, Tom. ``Online Anonymity in a Box, for \$49'', \emph{MIT Technology Review}, 21 November 2013. \url{http://www.technologyreview.com/news/521676/online-anonymity-in-a-box-for-49/}.

\bibitem{toubiana} Toubiana, V., Narayanan, A., Boneh, D., Nissenbaum, H., and Barocas, S. ``Adnostic: Privacy preserving targeted advertising.'' \emph{In Proceedings of the 2010 Network and Distributed System Security Symposium}, March 2010.

\bibitem{wired} Solon, Olivia. ``Safeplug makes it super-easy to harness Tor\'s anonymity at home,'' \emph{Wired UK}, 22 November 2013. \url{http://www.wired.co.uk/news/archive/2013-11/22/safeplug-tor}.

\bibitem{tormailinglist} Tor Mailing List.  \url{https://lists.torproject.org/pipermail/tor-talk/2013-November/031199.html}.

\bibitem{orbot} ``Tor on Android.'' \emph{Tor Project.} Accessed February 2014. \url{https://torproject.org/docs/android.html.en}

\bibitem{torproject} The Tor Project.  \url{https://www.torproject.org/}.

\bibitem{amorbot} The Tor Project. ``Orbot: Proxy with Tor.'' \emph{Google Android Market}. Accessed 7 February 2014, \url{https://play.google.com/store/apps/details?id=org.torproject.android}

\bibitem{wireshark} Wireshark, \url{http://www.wireshark.org/}. 

\end{thebibliography}


\end{document}
